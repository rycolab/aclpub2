It is a great honor for the Italian community of computational linguistics to host for the first time the Annual Meeting of ACL. Italy has an important place in the past of our field, and now hosts research institutions, groups, and companies that actively contribute to constantly advance the state of the art. The recent birth of the Italian Association of Computational Linguistics is the proof of how lively is the research on computational linguistics in our country.
Computational Linguistics is now experiencing an unprecedented success and popularity, as also testi- fied by the incredibly high number of participants to ACL 2019. Actually, the Florence edition is going to record the highest number of attendees to an ACL event so far. On the local side, this ever-growing number of participants has made the organization of ACL 2019 quite challenging: although not everything will be perfect, we did our best to offer the best conference experience to all participants.
We are quite enthusiastic of the conference venue, Fortezza da Basso: it is located in the hearth of Florence, it offers a large variety of rooms, both modern and “monumental”, it is in itself an “open air” museum, and it is also the ideal location for a very special social event on Tuesday, July 30.
This has been made possible thanks to the constant commitment and effort of the local organization teams: while we want to thank all the people involved, we are particularly grateful for the participation of researches and students from the Italian community, who really mobilized itself for ACL 2019. We would like to thank the people who were deeply involved in the local organization: the co-chairs of the Local Arrangement Committee (Sara Goggi and Maria Cristina Schiavone), the co-chairs of the Local Sponsorship (Roberto Basili and Giovanni Semeraro), and the Coordinators of the Student Volunteers (Dominique Brunato, Marco Senaldi, and Giulia Venturi), and of course the many volunteers, who are going to give a crucial help during the conference.
It was a very rewarding experience for us to work together with the team of experienced colleagues who coordinated the scientific organization of ACL 2019: thanks a lot to all of them for the continuous support! We also learned a lot from Priscilla Rasmussen, the ACL Business Manager, who always advised us on the best way to approach problems.
We also would like to thank our institutions, Università di Pisa, Istituto di Linguistica Computazionale “A. Zampolli” of CNR, and Fondazione Bruno Kessler, for the support they have given to the organization of ACL in Florence.
We thank Firenze Fiera for arranging the conference venue, and Firenze Convention \& Visitors Bureau for their help in arranging accommodations and for the organization of the social program.
Last but not least, our deepest thanks go to Maria Cristina Schiavone of MCI, for her professionalism, passion and effort, which have always allowed us to find the best possible solutions for the conference. Maria Cristina has successfully lead all the complex phases of the organization from the bidding until the smallest details in the daily structure of the event.
Enjoy your time at the ACL 2019! Benvenuti alla Fortezza da Basso, benvenuti a Firenze!

ACL 2019 Local Organization Co-Chairs
Alessandro Lenci, University of Pisa
Bernardo Magnini, Bruno Kessler Foundation
Simonetta Montemagni, Institute for Computational Linguistics of CNR